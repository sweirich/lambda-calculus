\documentclass{article}

\usepackage{amsmath,amssymb,amsthm}
\usepackage{ottalt}
\usepackage{mathpartir}
\usepackage{supertabular}
\usepackage{url}
% \usepackage[utf8x]{inputenc}

\usepackage{newunicodechar}
\usepackage{mathrsfs}

\usepackage{listings}
\usepackage{listingsutf8}

\usepackage{fancyvrb}
\usepackage[usenames,dvipsnames,svgnames,x11names]{xcolor}
%\usepackage{color}

\theoremstyle{definition}
\newtheorem{definition}{Definition}[section]
\newtheorem{lemma}{Lemma}[section]
\newtheorem{spec}{Specification}[section]

% for PDF latex, all unicode chars in this buffer must be first declared here
\DeclareUnicodeCharacter{2208}{\ensuremath{\in}}

% \inputott{lc-rules}

\title{Graph Model for the untyped lambda calculus and for 
Verse}

% lstlistings options
\usepackage{lstcoq}
\usepackage{lstparams}

% this pulls a listing from a Coq file.
\newcommand{\codeplus}[3]{%
  \lstinputlisting[%
        #3,
        linerange={#2},
        rangebeginprefix=(*\ begin\ ,%
        rangebeginsuffix=\ *),%
        rangeendprefix=(*\ end\ ,%
        rangeendsuffix=\ *),
        includerangemarker=false]{#1}}
\newcommand{\code}[2]{\codeplus{#1}{#2}{}}

%\newunicodechar{∈}{\ensuremath{\in}}


\begin{document}
\maketitle

This document describes a denotational semantics for a
based on a ``graph'' model. 

\newcommand\inhabited[1]{\ensuremath{\mathit{inhabited}}\ {#1}}
\newcommand\apply[2]{\ensuremath{{#1}\ \blacksquare\ {#2}}}

\section{Part one: untyped lambda calculus}


The file \texttt{simple/model.v} contains a sketch of this semantics for the
call-by-value untyped lambda calculus.

This semantics interprets lambda calculus functions as a graph: a set of
input-output pairs, each pair written $w \mapsto v$, where $w$ and $v$
are elements of some domain $\mathcal{W}$. 

The graph semantics also includes a special term $\circ$ as part of this set,
to signal function values that cannot be applied. See below.

\subsection{Examples}

Here are some examples.

\begin{itemize}

\item The graph of the increment function $\lambda x. x+1$ looks like

\[ \{ k \mapsto k+1\ |\ k \in \mathcal{N} \} \]

This graph maps natural numbers $k$ to their sucessors.

\item The graph of the identity function $\lambda x.x$ looks like this:

\[ \{ w \mapsto w\ |\ w \in \mathcal{W} \} \]

This graph maps all inputs $w$ to $w$. 

\item If we have a nonterminating expression, like $\omega$, i.e. 
  $(\lambda x.x)(\lambda x.x)$,
  then its denotation is the emptyset.  It is not
  a value and we cannot apply it to any arguments to get a results.

   \[ \{\} \]

\item The denotation of a function that takes an argument and then diverges
  i.e. $\lambda x. \omega$ is the special term only.

  \[ \{ \circ \} \]

  The term $\circ$ marks that this term is a value, but there are still no mappings
  in the graph. This term lets us distinguish the denotation of a diverging
  expression from that of a value, even if we cannot use the value.

\item The denotation of $K$, i.e. $\lambda x.\lambda y.x$ is  

  \[ \{ w_1 \mapsto \{ w_2 \mapsto w_1\ |\ w_2 \in \mathcal{W} \}\ |\ w_1 \in \mathcal{W} \} \]

\noindent but we will see below that this definition could also be 

  \[ \{ w_1 \mapsto (w_2 \mapsto w_1)\ |\ w_1, w_2 \in \mathcal{W} \} \]

because this set is extensionally equivalent. (SCW: need to define this equivalence).

\end{itemize}


\subsection{Semantic operations: specification}


What is this domain $\mathcal{W}$? Informally, we would like it to be the
powerset of all mappings (plus natural numbers and the trivial term $\circ$):

\[ \mathcal{W} = \mathcal{N} \cup \circ \cup \mathscr{P}( 
  \{ w_1 \mapsto w_2\ |\ w_1, w_2 \in \mathcal{W} \} )  \] 

\noindent But that is not a well founded definition. 
But, this is a specification of what we want the domain to look like.

To define a semantics for the untyped lambda calculus, we need to say how domain 
values can be constructed and used.

For that, we will specify application and abstraction operators,
$\blacksquare$ and $\Lambda$:

\begin{spec}[APPLY] For $w_1, w_2 \in \mathcal{W}$, we want
\[  \apply{w_1}{w_2}\ = \{ w\ |\ (w_2 \mapsto w) \in w_1 \} \]
\end{spec}

\begin{spec}[LAMBDA]
  Given $F \in \mathcal{W} \rightarrow \mathcal{W}$ and
  $w_1 \in \mathcal{W}$, we want
\[ \Lambda\ F = \{ \circ \} \cup \{ (w_1 \mapsto w_2)\ |\ w_2 \in F(w_1) \} \]
\end{spec}


\begin{spec}
Forall $F \in \mathcal{W} \rightarrow \mathcal{W}$, and 
  $w\in\mathcal{W}$, we want
  \[ \apply{\Lambda F}{w} = F\ w \]
\end{spec}

Now that we have apply, we can specify what it means for two elements of 
our domain to be equal. We want a relation that allows us to approximate 
the individual mappings. It shouldn't require that both sets have exactly 
the same graph, just that the two graphs can be used in the same way. 
\begin{spec}[Extensional Equality]
\[ w_1 \equiv w_2 = \inhabited{w_1} \leftrightarrow \inhabited{w_2}\ \wedge
  \forall w \in \mathcal{W}, \apply{w_1}{w} \equiv \apply{w_2}{w} \]
\end{spec}
  
For example, the specification above allows us to consider these two graphs as 
equivalent.

\[
\{ \{ 1 \mapsto (3 \mapsto 2), 1 \mapsto (4 \mapsto 5) \} \}
\equiv
\{ \{1 \mapsto \{ 3 \mapsto 2, 4 \mapsto 5 \} \}
\]

\subsection{Representing the domain}

So how do we represent $\mathcal{W}$ in a proof assistant?

We can represent the powerset of some type by its characteristic function,
i.e. a proposition that tells us whether the value is in the set.

\begin{lstlisting}{language=Coq}
Definition P (A : Type) := A -> Prop.
\end{lstlisting}

However, this definition does not give us an \emph{inductive} or finite
representation of values. For that we need to consider only finite sets of
values. (We'll use the type constructor \texttt{fset A} to refer to a finite
set containing elements of type \texttt{A}). This type can be inductively
defined and injected into the non-inductive variant \texttt{P A}. Using this
finite set type, we can build up our representation out of an inductive
representation of a mapping (\texttt{v\_map}), numbers (\texttt{v_nat}) and
trivial term (\texttt{v\_fun}):

\codeplus{../coq/simple/model.v}{Value}{language=Coq}

And then represent lambda calculus expressions to a potentially infinite set
of these terms.

\begin{lstlisting}{mathescape,language=Coq}
Definition W := P Value.
\end{lstlisting}

As an example, let's consider the identity function. The 
semantics of this function is:

\codeplus{../coq/simple/model.v}{idset}{} 

Consider how this definition reacts with the semantic function for application:

\begin{lstlisting}[language=Coq,mathescape]
Definition APPLY (D1 : P Value) (D2 : P Value) : P Value :=
  fun w $\Rightarrow$ exists V, (v_map V w $\in$ D1) $\wedge$ (mem V $\subseteq$ D2) $\wedge$ (valid_mem V).
\end{lstlisting}

Or, equally expressed as an inductive:

\codeplus{../coq/simple/model.v}{APPLY}{} 

\section{Part two: Extending the model to verse}

The files in the  \texttt{verse/} subdirectory contain the semantics of a much richer language.

\begin{itemize}
\item Verse contains multiple types of values: not only functions, but also integers and finite lists of values.
\item Because of the multiple types, there is the possibility
that the meaning of a term might be a \emph{run-time error}. For
example, if we try to apply the number 3 to itself.
\item Verse terms, if they don't produce an error, may also 
  produce multiple results, using choice.
\end{itemize}



\end{document}
